\documentclass[a4paper,12pt]{article}
\usepackage[utf8]{inputenc}
\usepackage[russian]{babel}
\usepackage{minted}
\usepackage{tcolorbox}
\usepackage{geometry}
\usepackage{xcolor}
\usepackage{titlesec}
\usepackage{tocloft}
\usepackage[hidelinks]{hyperref}


\geometry{top=2cm,bottom=2cm,left=2.5cm,right=2.5cm}

\definecolor{bg}{rgb}{0.95,0.95,0.95}
\definecolor{frame}{rgb}{0.3,0.3,0.3}
\setlength{\parskip}{0.7em}
\setlength{\parindent}{0pt}

\titleformat{\section}{\Large\bfseries}{\thesection}{1em}{}
\titleformat{\subsection}{\large\bfseries}{\thesubsection}{1em}{}

\renewcommand{\cftsecleader}{\cftdotfill{\cftdotsep}}

% Добавление стандартного заголовка LaTeX
\title{Документация ParaCL}
\date{}

\begin{document}

\maketitle % Вывод заголовка

\tableofcontents
\newpage

\section{Введение}

Язык \textbf{ParaCL} — это простой язык программирования с поддержкой арифметики, ввода, вывода, условий, циклов, массивов и функций.

\section{Базовые конструкции}

В ParaCL доступны стандартные арифметические операции, условные конструкции \texttt{if} и циклы \texttt{while}, аналогично языку программирования C.

\section{Типы данных}
ParaCL поддерживает следующие типы данных:
\begin{itemize}
    \item Целые числа
    \item Массивы
    \item Функции
\end{itemize}

\section{Массивами}

Массив — структура данных для хранения однотипных элементов. Существует два вида массивов: статические и динамические.

\subsection{Статические массивы}

Если размер массива известен на этапе компиляции, он размещается на стеке:

\begin{tcolorbox}[colback=bg, colframe=frame, title=Пример статического массива]
\begin{minted}[fontsize=\small, bgcolor=bg]{text}
v0 = 42;
v1 = repeat(v0, 5);  // Массив из 5 элементов со значением 42
\end{minted}
\end{tcolorbox}

\textbf{Индексация} осуществляется через квадратные скобки:

\begin{tcolorbox}[colback=bg, colframe=frame, title=Индексирование массива]
\begin{minted}[fontsize=\small, bgcolor=bg]{text}
v3 = 3;
v4 = v1[2] + v1[v3];
v5 = repeat(v4, v3);  // Новый массив из 3 элементов
\end{minted}
\end{tcolorbox}

\subsection{Динамические массивы}

Если размер массива неизвестен заранее, создаётся динамический массив (в куче). Он освобождается автоматически по выходу из области видимости:

\begin{tcolorbox}[colback=bg, colframe=frame, title=Пример динамического массива]
\begin{minted}[fontsize=\small, bgcolor=bg]{text}
v7 = ?;
v8 = repeat(2, v7);  // Массив из v7 элементов со значением 2
\end{minted}
\end{tcolorbox}

Для создания неинициализированного массива используется ключевое слово \texttt{undef}:

\begin{tcolorbox}[colback=bg, colframe=frame, title=Неинициализированный массив]
\begin{minted}[fontsize=\small, bgcolor=bg]{text}
v9 = repeat(undef, ?);  // Размер читается с stdin
\end{minted}
\end{tcolorbox}

\subsection{Массивы массивов и инициализация}

Массивы могут содержать другие массивы:

\begin{tcolorbox}[colback=bg, colframe=frame, title=Массив массивов]
\begin{minted}[fontsize=\small, bgcolor=bg]{text}
v10 = repeat(v9, ?);
\end{minted}
\end{tcolorbox}

Для явной инициализации применяется ключевое слово \texttt{array}:

\begin{tcolorbox}[colback=bg, colframe=frame, title=Пример инициализации]
\begin{minted}[fontsize=\small, bgcolor=bg]{text}
v11 = array(1, 2, 3, 4, 5);
\end{minted}
\end{tcolorbox}

Частичная инициализация и вложенные \texttt{repeat} также поддерживаются:

\begin{tcolorbox}[colback=bg, colframe=frame, title=Частичная инициализация с flatten]
\begin{minted}[fontsize=\small, bgcolor=bg]{text}
v12 = array(undef, 1, 2, 3, 4, 5, repeat(undef, 5));
v13 = array(repeat(1, 5), repeat(2, 5));
// Результат: [1,1,1,1,1,2,2,2,2,2] (одномерный)
\end{minted}
\end{tcolorbox}

\section{Функции и области видимости}

В ParaCL функции — полноценные объекты.

\subsection{Объявление функции}
Функции определяются с помощью ключевого слова \texttt{func}. 

\begin{tcolorbox}[colback=bg, colframe=frame, title=Возврат последнего выражения]
\begin{minted}[fontsize=\small, bgcolor=bg]{text}
bar = func() {
  x = 5;
  y = 10;
  x + y;
}; // точка с запятой здесь возможна но необязательна
t = bar(); // t == 15
\end{minted}
\end{tcolorbox}

\subsection{Явный return}

\begin{tcolorbox}[colback=bg, colframe=frame, title=Возвращаемое значение с return]
\begin{minted}[fontsize=\small, bgcolor=bg]{text}
buz = func(x) {
  if (x != 5)
    return x;
  y = 10;
  x + y;
}
z = buz(6); // z == 6
\end{minted}
\end{tcolorbox}

\subsection{Значение блока}

Любой блок может возвращать значение, также блок захватывает внешние переменные:

\begin{tcolorbox}[colback=bg, colframe=frame, title=Блок как выражение]
\begin{minted}[fontsize=\small, bgcolor=bg]{text}
t = { x = 5; y = 10; x + y; } // t == 15
\end{minted}
\end{tcolorbox}

\subsection{Именованные функции и рекурсия}

У функции может быть имя, если оно нужно (например для рекурсии) тогда оно записывается через двоеточие:

\begin{tcolorbox}[colback=bg, colframe=frame, title=Пример рекурсивной функции]
\begin{minted}[fontsize=\small, bgcolor=bg]{text}
fact = func(x) : factorial {
  res = 1;
  if (x > 0)
    res = factorial(x - 1);
  res;
}
\end{minted}
\end{tcolorbox}

\end{document}